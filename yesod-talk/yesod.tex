\documentclass{beamer}

\mode<presentation>
{
  \usetheme{Warsaw}
  % or ...

  \setbeamercovered{transparent}
  % or whatever (possibly just delete it)
}

\usepackage{graphicx}
\usepackage{epstopdf}
\usepackage{listings}
\usepackage{hyperref}
\usepackage{soul}
\usepackage[english]{babel}
\lstset{language=Haskell}
% or whatever

\usepackage[latin1]{inputenc}
% or whatever

\usepackage{times}
\usepackage[T1]{fontenc}
% Or whatever. Note that the encoding and the font should match. If T1
% does not look nice, try deleting the line with the fontenc.


\title[Web programming in Haskell using Yesod] % (optional, use only with long paper titles)
{Web Programming in Haskell using Yesod }

\author[Sibi Prabakaran] % (optional, use only with lots of authors)
{}
% - Use the \inst{?} command only if the authors have different
%   affiliation.

\institute[Bangalore Haskell User Group Meetup] % (optional, but mostly needed)
{
  Bangalore Haskell User Group Meetup \\
  Xebia India
  %% TCS Innovation Labs \\
  %% Cyber Physical Systems Lab
}
% - Use the \inst command only if there are several affiliations.
% - Keep it simple, no one is interested in your street address.

\date[] % (optional)
{Dec 16, 2017}
%% {\today}

\subject{Talks}
% This is only inserted into the PDF information catalog. Can be left
% out. 

% If you have a file called "university-logo-filename.xxx", where xxx
% is a graphic format that can be processed by latex or pdflatex,
% resp., then you can add a logo as follows:

% \pgfdeclareimage[height=0.5cm]{university-logo}{university-logo-filename}
% \logo{\pgfuseimage{university-logo}}



% Delete this, if you do not want the table of contents to pop up at
% the beginning of each subsection:
\AtBeginSubsection[]
{
  \begin{frame}<beamer>{Outline}
    \tableofcontents[currentsection,currentsubsection]
  \end{frame}
}


% If you wish to uncover everything in a step-wise fashion, uncomment
% the following command: 

%% \beamerdefaultoverlayspecification{<+->}

\lstset{language=Haskell, showstringspaces=false}

\begin{document}

\begin{frame}
  \titlepage
\end{frame}

\begin{frame}{About Me}
\begin{itemize}
\item
Sibi
\item
Consultant for Xebia
\item
Internet Handle: \href{http://psibi.in/about.html}{\beamergotobutton{psibi}}
\item
  Mail: sibi@psibi.in
\item \href{http://psibi.in}{\beamergotobutton{http://psibi.in}}
\end{itemize}
\end{frame}

\begin{frame}{Outline}
  \tableofcontents
  % You might wish to add the option [pausesections]
\end{frame}


% Since this a solution template for a generic talk, very little can
% be said about how it should be structured. However, the talk length
% of between 15min and 45min and the theme suggest that you stick to
% the following rules:  

% - Exactly two or three sections (other than the summary).
% - At *most* three subsections per section.
% - Talk about 30s to 2min per frame. So there should be between about
%   15 and 30 frames, all told.

\section{Introduction}

\begin{frame}{Goal}
\begin{itemize}

\item Viable choice
\item Yesod's philosophy
\item Survey of yesod related libraries, ORM backends etc.
\item Real world experience

\end{itemize}
\end{frame}

\begin{frame}{History}
  % - A title should summarize the slide in an understandable fashion
  %   for anyone how does not follow everything on the slide itself.
  \begin{itemize}
  \item Yesod is a Hebrew word
  \item Originally developed by Michael Snoyman
  \item Initial release: March 8, 2010
  \item License: MIT
  \end{itemize}
\end{frame}

\begin{frame}{Why Yesod?}
  \begin{itemize}
  \item \st{R} evolutionary framework
  \item Avoid bugs at compile time
  \item DSLs for common tasks
  \item Performant
  \end{itemize}
\end{frame}

\section{MVC layer}

\begin{frame}{Persistent}

  \begin{itemize}
  \item Database agnostic
  \item Data modeling
  \item Migrations
  \end{itemize}
\end{frame}

\begin{frame}[fragile]{Declare tables}
\begin{lstlisting}
  mkPersist [persist|
  User
    name String
    age Int
  Todo
    userId UserId  
    task Text
    done Bool
  |]
\end{lstlisting}
\begin{itemize}
\item Other syntax: sqltype, constraint, sql etc.
\item Deriving json
\item Reference: \href{https://github.com/yesodweb/persistent/blob/master/docs/Persistent-entity-syntax.md}{\beamergotobutton{Persisent docs}}
\end{itemize}

\end{frame}

\begin{frame}[fragile]{CRUD}
  \begin{lstlisting}
    simonId <- insert $ "SPJ" 34
    marlowId <- insert $ "Marlow" 35

    insert_  $ simonId "Buy dragon book" False

    simonTasks <- selectList
                  [TodoUserId ==. simonId]
                  [LimitTo 5]

    (simon :: Maybe User) <- get simonId

    delete marlowId
    deleteWhere [TodoUserId ==. simonId]
  \end{lstlisting}
\end{frame}

\begin{frame}{Routing}
  \begin{itemize}
  \item Single data type for representing URLs
  \item Mapping between route to handler is done
  \item Two way parsing functions created
  \item Sync between parsing and handler functions ensured
  \end{itemize}
\end{frame}

\begin{frame}[fragile]{Route syntax}

  \begin{lstlisting}
/ HomeR GET POST
/hello HelloR
/fib/#Int FibR GET    
\end{lstlisting}

\begin{itemize}
\item Canonical URLs (joinPath, cleanPath)
\item Pieces: Static, Dynamic single, Dynamic multi  
\end{itemize}

\end{frame}

\begin{frame}[fragile]{Routing internals}
\begin{lstlisting}
  data MyAppRoute = HomeR | HelloR | FibR Int

  renderMyAppRoute HomeR = []
  renderMyAppRoute HelloR = ["hello"]
  renderMyAppRoute (FibR int) =
      ["fib", toSinglePiece int]

  parseMyAppRoute [] = Just HomeR
  parseMyAppRoute ["hello"] = Just HelloR
  parseMyAppRoute ["fib", int] = do
      fibInt <- fromSinglePiece int
      return $ FibR fibInt
  parseMyAppRoute _ = Nothing
\end{lstlisting}
\end{frame}

\begin{frame}{Template languages}
  \begin{itemize}
  \item DSLs for templating
  \item Compile time syntax checked
  \item Variable interpolation
  \item Control structures for Hamlet
  \end{itemize}
\end{frame}

\begin{frame}[fragile]{Hamlet}
\begin{lstlisting}[language=html,basicstyle=\small]
$doctype 5
<html>
  <head>
    <title>#{pageTitle} - My Site
      <link rel=stylesheet href=@{StylesheetR}>
    <body>
      <h1 .page-title>#{pageTitle}
      <p>Here is a list of your friends:
      $if null friends
         <p>Sorry, I lied
         <p>You don't have any friends.
      $else
         <ul>
            $forall Friend name age <- friends
              <li>#{name} (#{age} years old)
      <footer>^{copyright}
\end{lstlisting}

\end{frame}
\begin{frame}[fragile]{Lucius \& Cassius (CSS)}
  Lucius
  \begin{lstlisting}[language=html,basicstyle=\small]
section.blog {
    padding: 1em;
    border: 1px solid #000;
    h1 {
        color: #{headingColor};
        background-image: url(@{MyBackgroundR});
    }
}
\end{lstlisting}
Cassius
\begin{lstlisting}[language=html,basicstyle=\small]
section.blog
    padding: 1em
    border: 1px solid #000
    h1
        color: #{headingColor}
        background-image: url(@{MyBackgroundR})  
  \end{lstlisting}
\end{frame}

\begin{frame}[fragile]{Julius (Javascript)}
  \begin{lstlisting}
  $(function(){
    $("section.#{sectionClass}").hide();
    $("#mybutton").click(function(){
      document.location = "@{SomeRouteR}";
    });
    ^{addBling}
  });
  \end{lstlisting}
\end{frame}

\begin{frame}[fragile]{XSS Protection}
\begin{lstlisting}[language=haskell]
name :: Text
name = "Sibi <script>alert('injected')</script>"

main :: IO ()
main = putStrLn $ renderHtml [shamlet|#{name}|]
\end{lstlisting}

Output:
\begin{lstlisting}[basicstyle=\tiny]
Sibi &lt;script&gt;alert(&#39;injected&#39;)&lt;/script&gt;
\end{lstlisting}
\end{frame}

\begin{frame}{XSS Protection}
  \begin{itemize}
  \item blaze-html package
  \item ToMarkup typeclass
  \item Textual values are always escaped (see ToMarkup instances)
  \item Html values aren't escaped
  \item preEscapedToHtml
  \end{itemize}
\end{frame}

\begin{frame}{Widgets}
\begin{itemize}
\item Glue between template languages
\item Ability to re-use a single UI component
\item Can perform IO operations (DB queries etc.)
\item Controls the generation of end HTML (<body>,<head> tags)
\end{itemize}
\end{frame}

\begin{frame}[fragile]{Widget: Example}
  \begin{lstlisting}[language=haskell,basicstyle=\tiny]
getHomeR = defaultLayout $ do
    setTitle "My Page Title"
    toWidget [lucius| h1 { color: green; } |]
    addScriptRemote "https://ajax.googleapis.com/ajax/libs/jquery/1.6.2/jquery.min.js"
    toWidget
        [julius|
            $(function() {
                $("h1").click(function(){
                    alert("You clicked on the heading!");
                });
            });
        |]
    toWidgetHead
        [hamlet|
            <meta name=keywords content="some sample keywords">
        |]
    toWidget
        [hamlet|
            <h1>Here's one way of including content
        |]
    [whamlet|<h2>Here's another |]
    toWidgetBody
        [julius|
            alert("This is included in the body itself");
        |]  
\end{lstlisting}
\end{frame}

\section{Deployment}

\begin{frame}{Deploying Yesod apps}
  \begin{itemize}
  \item Common advice: Don't compile on server
  \item Change config file for production (Personally used CPP)
  \item Files to deploy: executable, static folder
  \item Keter
  \item Hapistrano
  \end{itemize}
\end{frame}

\begin{frame}{Other parts}
  \begin{itemize}
\item Subsites
\item Middlewares
\item clientsession
\item yesod-form
\item yesod-auth
\item yesod-fb
\item yesod-sitemap  
\item websocket/eventsource support
\item Lots of others in Hackage
\item Real world experience
  \end{itemize}
\end{frame}

\begin{frame}{Want to contribute?}
  \begin{itemize}
  \item Github: psibi/yesod-rest \href{https://github.com/psibi/yesod-rest}{\beamergotobutton{yesod-rest}}
  \item Github: psibi/wai-slack-middleware \href{https://github.com/psibi/wai-slack-middleware}{\beamergotobutton{wai-slack}}
  \item yesod, persistent etc.
  \end{itemize}

  \hfill \break
  \hfill \break
  \hfill \break
  \hfill \break
  Questions... ?
\end{frame}


\end{document}
